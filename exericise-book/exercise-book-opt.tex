\documentclass[12pt]{article}

\usepackage[utf8]{inputenc}
\usepackage[T2A]{fontenc}
\usepackage[english, russian]{babel}
\usepackage{amsmath, amsthm, amssymb}
\usepackage{enumerate,hhline, bm}
\usepackage[mathscr]{eucal}
%\usepackage{mathtext}

\usepackage{hyperref}
\hypersetup{unicode=true, final=true, colorlinks=true}


\input{optimization-defs.tex}

\theoremstyle{remark}
\newtheorem{exercise}{}[subsection]
\renewcommand{\theexercise}{\textbf{\textnumero \arabic{exercise}}}

%\DeclareMathOperator{\cov}{cov}
%\DeclareMathOperator{\corr}{corr}
%\DeclareMathOperator{\Var}{Var}

%\topmargin=-2cm%-1.5cm

%\addtolength{\textheight}{3cm}

%\oddsidemargin=-0.1cm

%\addtolength{\textwidth}{1.8cm}


\title{Задачи по Методам оптимизации и Теории игр}
\author{Артамонов Н.В.}
%\date{весна 2014}

%\title{Задачи для подготовки к экзамену по курсу 
%<<Методы оптимальных решений>>}\author{\copyright Артамонов Н.В., кафедра ЭММАЭ}

\begin{document}

\maketitle

%\markright{}
\tableofcontents

\section{Задачи оптимизации}

\textbf{Внимание: Во всех расчетных задачах обязательно проверять достаточные условия экстремума!
Задачи, отмеченные звёздочками, не является обязательными}

\subsection{Безусловная оптимизация}

% !TEX root = exercise-book-opt.tex

\begin{exercise}
Найдите локальные экстремумы функций
\begin{align*}
	f(x,y) &= 10-6x-4y+2x^2+y^2-2xy \\
	f(x,y) &= 8+8x+4y-5x^2-2y^2+6xy \\
	f(x,y) &= 5+2x+6y+5x^2+3y^2+8xy
\end{align*}
% Нарисуйте графики функций (MS Excel, Python etc)
\end{exercise}

\begin{exercise}
Найдите локальные экстремумы функций
\begin{align*}
	f(x,y,z) &= 6+4x+2y+6z+2x^2+2y^2+z^2+2xy+2yz \\
	f(x,y,z) &= 3+4x+8y+4z-3x^3-2y^2-4z^2+2xy+2xz+4yz\\
	f(x,y,z) &= 8+2x+4y+2z+2x^2+y^2+3z^2+2xy+4xz+4yz
\end{align*}
\end{exercise}

\begin{exercise}
Найдите локальные экстремумы функций
\begin{align*}
	f(x,y) &= 5+x^3-y^3+3xy \\
	f(x,y) &= 3x^2y+y^3-3x^2-3y^2+2 \\
	f(x,y) &= x^3+x^2y-2y^3+6y
\end{align*}
\end{exercise}

\begin{exercise}
Завод производит три вида товаров и продает их по ценам $P_1=2$, $P_2=4$ и $P_3=6$.
Издержки производства равны
\[
	C(Q_1,Q_2,Q_3)=2Q_1^2+Q_2^2+2Q_3^2-2Q_2Q_3
\]
($Q_1, Q_2, Q_3$ -- объемы производства товаров). Найдите оптимальные объемы производства. 
% Какой экономической ситуации соответствует экзогенность цен?
\end{exercise}

\begin{exercise}
Завод производит три вида товаров и продает их по ценам $P_1=2$, $P_2=2$ и $P_3=3$.
Издержки производства равны 
\[
	C(Q_1,Q_2,Q_3)=2Q_1^2+Q_2^2+2Q_3^2-2Q_2Q_3-2Q1Q_3
\]
($Q_1, Q_2, Q_3$ -- объемы производства товаров).
Найдите оптимальные объемы производства.
\end{exercise}

\begin{exercise}
Завод производит два вида товаров, (обратные) функции спроса на которые
имеют вид $P_1=21-5Q_1+2Q_2$ и $P_2=35-Q_2+2Q_1$. Функция
издержек равна
\[
	C(Q_1,Q_2)=Q_1+3Q_2
\]
($Q_1, Q_2$ -- объемы производства товаров). Найдите оптимальные объемы производства. 
% Какой экономической ситуации соответствует эндогенность цен?
\end{exercise}

\begin{exercise}
Завод производит два вида товаров, (обратные) функции спроса на которые
имеют вид $P_1=51-2Q_1+3Q_2$ и $P_2=47-5Q_2+3Q_1$. Функция
издержек равна
\[
	C(Q_1,Q_2)=3Q_1+5Q_2
\]
($Q_1, Q_2$ -- объемы производства товаров). Найдите оптимальные объемы производства. 
% Какой экономической ситуации соответствует эндогенность цен?
\end{exercise}

\begin{exercise}
Найдите локальные экстремумы функций
\begin{align*}
	f(x,y) &= 6\ln x+8\ln y-3x-2y \\
	f(x,y) &= 4\ln x+6\ln y+2x-3xy \\
	f(x,y) &= 5\ln x+4\ln y-x-4xy
\end{align*}
\end{exercise}

\subsection{Выпуклость}

% !TEX root = exercise-book-opt.tex

\begin{exercise}
Исследуйте на выпуклость/вогнутость функции на \(\R^2\)
\begin{align*}
	f(x,y) &= 10-6x-4y+2x^2+y^2-2xy \\
	f(x,y) &= 8+8x+4y-5x^2-2y^2+6xy \\
	f(x,y) &= 5+2x+6y+5x^2+3y^2+8xy \\
	f(x,y) &= 10+x^2+y^2+2xy \\
	f(x,y) &= 5+4xy-2x^2-2y^2
\end{align*}
\end{exercise}

\begin{exercise}
Исследуйте на выпуклость/вогнутость функции на \(\R^3\)
\begin{align*}
	f(x,y,z) &= 6+4x+2y+6z+2x^2+2y^2+z^2+2xy+2yz \\
	f(x,y,z) &= 3+4x+8y+4z-3x^3-2y^2-4z^2+2xy+2xz+4yz\\
	f(x,y,z) &= 8+2x+4y+2z+2x^2+y^2+3z^2+2xy+4xz+4yz
\end{align*}
\end{exercise}

\begin{exercise}
При каких значениях параметра $\beta$ функция
\[
	f(x_1,x_2,x_3)=2x_1^2+4x_2^2+x_3^2-\beta x_1x_3
\]
будет строго/нестрого выпуклой? Строго/нестрого вогнутой?
\end{exercise}

\begin{exercise}
Исследуйте на выпуклость/вогнутость функции, определённые на 
\(\R^2_+\) (\(a,b>0\))
\begin{align*}
	f(x,y) &= a\ln x+b\ln y-2x^2-y^2-2xy \\
	f(x,y) &= x^2+5y^2-4xy-a\ln x-b\ln y \\
	f(x,y) &= a\ln x+b\ln y -3x^2-5y^2-8xy
\end{align*}
\end{exercise}

\subsection{Оптимизации с ограничениями  равенства}

% !TEX root = exercise-book-opt.tex

\begin{exercise}
Решите задачи оптимизации
\begin{align*}
	&\begin{gathered}
		\max(2x+3y) \\ s.t.\; 2x^2+y^2=11 
	\end{gathered} &
	&\begin{gathered}
		\min(2x+3y) \\ s.t.\; 2x^2+y^2=11 
	\end{gathered} \\
	&\begin{gathered}
		\max(5x-3y) \\ s.t.\; x^2+3y^2=28 
	\end{gathered} &
	&\begin{gathered}
		\min(5x-3y) \\ s.t.\; x^2+3y^2=28 
	\end{gathered}
\end{align*}
\end{exercise}

\begin{exercise}
Решите задачи оптимизации
\begin{align*}
	& \begin{gathered}
		\min(x^2+2y^2) \\ s.t.\;3x+2y=22
	\end{gathered} &
	& \begin{gathered}
		\max(10-2x^2-18y^2) \\ s.t.\;4x+6y=30
	\end{gathered} \\
	& \begin{gathered}
		\max(y^2-2x^2) \\ s.t.\;4x+3y=5
	\end{gathered} &
	& \begin{gathered}
		\min(2y^2-x^2) \\ s.t.\;5x+4y=17
	\end{gathered}
\end{align*}
\end{exercise}

\begin{exercise}
Решите задачи оптимизации
\begin{align*}
	& \begin{gathered}
		\max(x^2y^2) \\ s.t.\;3x+2y=24
	\end{gathered} &
	& \begin{gathered}
		\min(x^2y^2) \\ s.t.\;3x+2y=24
	\end{gathered}
\end{align*}
\end{exercise}

\begin{exercise}
Решите задачи оптимизации
\begin{align*}
	& \begin{gathered}
		\max(xy) \\ s.t.\;x^2+2y^2=36
	\end{gathered} &
	& \begin{gathered}
		\min(xy) \\ s.t.\;x^2+2y^2=36
	\end{gathered}
\end{align*}
\end{exercise}

\begin{exercise}
Найти экстремум функции полезности $u=x^2y$ при бюджетном ограничении
$2x+3y=90$. 
% Дайте экономическую интерпретацию параметров функции полезности.
\end{exercise}

\begin{exercise}
Для производства предприятие закупает два вида ресурсов по ценам
$P_1=10$ и $P_2=20$, бюджет составляет \$1200. Производственная
функция предприятия равна $f(x,y)=\sqrt{xy}$. 
Найдите количество ресурсов с целью обеспечения оптимальной производственной программы.
% Дайте экономическую интерпретацию производственной функции и ее параметров.
\end{exercise}

\begin{exercise}
Для производства предприятие закупает два вида ресурсов по ценам
$P_x=5$ и $P_y=2$, бюджет составляет \$200. Производственная
функция предприятия равна $f(x,y)=2\sqrt{xy}$.
Найдите количество ресурсов с целью обеспечения оптимальной производственной программы.
% \begin{enumerate}
% 	\item Какой экономической ситуации соответствует экзогенность цен?
% 	\item  Будет ли производственная функция однородной? Если да, то какой степени и дайте
% 	интерпретацию степени однородности.
% 	\item Постройте модель для нахождения оптимального производства. %производственной программы.
% 	\item Приведите необходимые условия экстремума.
% 	\item Приведите достаточные условия экстремума.
% 	\item Какое количество ресурсов необходимо закупить?
% 	\item Дайте экономическую интерпретацию множителя Лагранжа.
% \end{enumerate}
\end{exercise}

\begin{exercise}
Производственная функция предприятия равна $f(x,y)=\sqrt{xy}$. Ресурсы
закупаются по ценам $P_1$ и $P_2$. Рассмотрим задачу оптимизации
\begin{gather*}
	\min (P_1x+P_2 y) \\  f(x,y)=Q_0
\end{gather*}
Дайте интерпретацию оптимальной задачи и найдите её решение.
% \begin{enumerate}
% 	\item Дайте интерпретацию экстремальной задачи с ограничениями
% 	\item Напишите функцию Лагранжа и необходимые условия
% 	экстремума.
% 	\item Сформулируйте достаточные условия экстремума.
% 	\item Найдите решения экстремальной задачи.
% 	\item Дайте экономическую интерпретацию множителя Лагранжа.
% 	Как (экономически) можно объяснить, что множитель Лагранжа не зависит то
% 	объема выпуска $Q_0$?
% \end{enumerate}  
\end{exercise}

\begin{exercise}
Потребительская корзина состоит их трех товаров, цена на которые равны
$P_1$, $P_2$, $P_3$. Доход равен $I$.  Функция полезности потребителя равна
\[
	U(q_1,q_2,q_3)=\ln q_1+\ln q_2+\ln q_3.
\]
Найдите оптимальную потребительскую корзину.
% \begin{enumerate}
% 	\item Постройте модель оптимизации для нахождения оптимальной
% 	потребительской корзины.
% 	\item Сформулируйте необходимые и достаточные условия экстремума.
% 	\item Найдите оптимальную потребительскую корзину.
% 	\item Дайте экономическую интерпретацию множителя Лагранжа.
% \end{enumerate}
\end{exercise}

\begin{exercise}
В условиях предыдущей задачи рассмотрите функцию полезности
\begin{align*}
	U(q_1,q_2,q_3)&=a\ln q_1+b\ln q_2+c\ln q_3 &
	a,b,c&>0
\end{align*}
\end{exercise}

\begin{exercise}
Фирма для производства использует два фактора производства: капитал и труд.
Производственная функция имеет вид $F=3KL^2$. Фирма решает следующую задачу
\begin{gather*}
	\min(5K+4L) \\ F(K,L)=9600
\end{gather*}
Дайте интерпретацию оптимальной задачи и найдите её решение. 
% Дайте интерпретацию множителям Лагранжа.
\end{exercise}

\begin{exercise}
Решите задачи оптимизации
\begin{gather*}
	\min(2x^2+2y^2+4z^2+2xy+2xz+2yz-10x-50y-10z) \\ 
	s.t.\;x+2y+3z=20 \\
	\max(10-9x-3y+3z-4x^2-2y^2-2z^2+4xy+2xz+2yz) \\
	s.t.\; 2x+2y-4z=7
\end{gather*}
\end{exercise}

\begin{exercise}
Решите \textbf{численно}\footnote{MS Excel/Python} задачи оптимизации
\begin{align*}
	&\max (x+y+z) & &\min(x^2+y^2+z^2) \\
	s.t.&\left\{\begin{aligned}
		2x^2+y^2+z^2 &= 9 \\ x-y+z&=0 
	\end{aligned}\right. &
	s.t.&\left\{\begin{aligned}
		2x+y+2z &= 10 \\ 3x-2y+z&=6 
	\end{aligned}\right. \\
	&\min (x+y+z) & &\min(x^2+y^2+z^2) \\
	s.t.&\left\{\begin{aligned}
		x^2+2y^2+2z^2 &= 16 \\ x+y-z&=0 
	\end{aligned}\right. &
	s.t.&\left\{\begin{aligned}
		2x^2+4y^2+3z^2 &= 16 \\ x+y-z&=0 
	\end{aligned}\right.
\end{align*}
\end{exercise}

\subsection{Оптимизация с ограничениями неравенства}

% !TEX root = exercise-book-opt.tex

\begin{exercise}
Решите задачи оптимизации
\begin{align*}
	&\begin{gathered}
		\min(2x^2+3y^2) \\  s.t\; x-y\geq 10
	\end{gathered} &
	&\begin{gathered}
		\max (2x+3y) \\ s.t.\; 5x^2+y^2+4xy\leq 25
	\end{gathered}
\end{align*}
\end{exercise}

\begin{exercise}
Решите задачи оптимизации
\begin{align*}
	&\begin{gathered}
		\max(2x+4y-5x^2-y^2-4xy)\\ s.t.\; 2x+3y\leq 40
	\end{gathered} &
	&\begin{gathered}
		\max(4x+8y-6x^2-3y^2-8xy)\\ s.t.\; 2x+2y\leq 9
	\end{gathered} \\
	&\begin{gathered}
		\max(4x+10y-2x^2-6y^2-6xy)\\ s.t.\; 3x+6y\leq 3
	\end{gathered} &
	&\begin{gathered}
		\max(8x+8y-3x^2-3y^2-6xy)\\ s.t.\; 2x+4y\leq 13
	\end{gathered}
\end{align*}
\end{exercise}

\begin{exercise}
Решите задачи оптимизации
\begin{align*}
	&\begin{gathered}
		\min(5x^2+y^2+4xy-2x-4y)\\ s.t.\; 2x+y\geq 40
	\end{gathered} &
	&\begin{gathered}
		\min(5x^2+y^2-4xy-2x-y)\\ s.t.\; 2x+3y\geq 10
	\end{gathered} \\
	&\begin{gathered}
		\min(5x^2+4y^2-8xy-3x-4y)\\ s.t.\; 2x+4y\geq 31
	\end{gathered} &
	&\begin{gathered}
		\min(2x^2+6y^2-6xy-4x-43)\\ s.t.\; 3x+6y\geq 20
	\end{gathered}
\end{align*}
\end{exercise}

\begin{exercise}
Решите задачи оптимизации
\begin{align*}
	& \begin{gathered}
		\max (x-2y) \\ s.t.\; x^2+y^2\leq 4
	\end{gathered} &
	& \begin{gathered}
		\max(1-(x+1)^2-(y-1)^2) \\ s.t.\; x+y\leq 10
	\end{gathered} \\
	& \begin{gathered}
		\max (x-2y) \\ s.t.\left\{\begin{aligned}
			 x^2+y^2&\leq 4 \\  x,y&\geq0
		\end{aligned}\right.
	\end{gathered} &
	& \begin{gathered}
		\max(1-(x+1)^2-(y-1)^2) \\ s.t.\left\{\begin{aligned}
			x+y&\leq 10  \\  x,y&\geq0
		\end{aligned}\right.
	\end{gathered}
\end{align*}
\end{exercise}

\begin{exercise}
Завод производит два вида товаров, цена на которые равны 
$P_1=10$ и $P_2=30$. Функция издержек равна
\[
	C(Q_1,Q_2)=Q_1^2+5Q_2^2-4Q_1Q_2
\]
($Q_1, Q_2$ -- объемы производства товаров). Найдите объемы производства,
оптимизирующие прибыль, если 
\begin{enumerate}
	\item издержки не должны превышать 104
	\item издержки не должны превышать 5850
\end{enumerate}
\end{exercise}

\begin{exercise}
Завод производит два вида товаров, (обратные) функции спроса на которые
имеют вид $P_1=20-2Q_1+3Q_2$ и $P_2=30-5Q_2+3Q_1$ (цены эндогенны). Функция
издержек равна $C(Q_1,Q_2)=10Q_1+20Q_2$
($Q_1, Q_2$ -- объемы производства товаров).  Найдите объемы производства,
оптимизирующие прибыль, если 
\begin{enumerate}
	\item издержки не должны превышать 3600
	\item издержки не должны превышать 450
\end{enumerate}
\end{exercise}

\begin{exercise}
Решите \textbf{численно}\footnote{MS Excel/Python} задачи оптимизации
\begin{align*}
	&\max (x-y+z) & &\max (x-y+z) \\
	s.t.\;& 2x^2+y^2+z^2\leq 9 &
	s.t.&\left\{\begin{aligned}
		2x^2+y^2+z^2 &\leq 9 \\ x,y,z&\geq 0 
	\end{aligned}\right. \\
	&\min (x^2+y^2+z^2) & &\min(x^2+y^2+z^2) \\
	s.t.&\left\{\begin{aligned}
		x+y-z &\geq 10 \\ x-2y+2z&\geq 8 
	\end{aligned}\right. &
	s.t.&\left\{\begin{aligned}
		x+y-z &\geq 10 \\ x-2y+2z&\geq 8 \\ x,y,z &\geq0 
	\end{aligned}\right.
\end{align*}
\end{exercise}

\begin{exercise}
Решите \textbf{численно}\footnote{MS Excel/Python} задачи оптимизации
\begin{align*}
	&\max (x_1-x_2-x_3+x_4) & &\max (x_1-x_2-x_3+x_4) \\
	s.t.\;& x_1^2+2x_2^2+x_3^2+3x_4^2\leq 25 &
	s.t.&\left\{\begin{aligned}
		2x_1^2+x_2^2+x_3^2+3x_4^2 &\leq 25 \\ x_1,x_2,x_3,x_4&\geq 0 
	\end{aligned}\right. \\
	&\min (x_1^2+x_2^2+x_3^2+x_4^3) & &\min(x_1^2+x_2^2+x_3^2+x_4^3) \\
	s.t.&\left\{\begin{aligned}
		x_1x_2-x_3+x_4 &\geq 8 \\ x_1+2x_2-2x_3-2x_4&\geq 5 
	\end{aligned}\right. &
	s.t.&\left\{\begin{aligned}
		x_1x_2-x_3+x_4 &\geq 8 \\ x_1+2x_2-2x_3-2x_4&\geq 5 \\ 
		x_1,x_2,x_3,x_4 &\geq0 
	\end{aligned}\right.
\end{align*}
\end{exercise}

\begin{exercise}[*]
Потребительская корзина состоит из двух товаров, её функция
полезности равна $U(x,y)=x+a\ln(y)$ (параметр $a>0$).
Потребитель решает оптимальную задачу
\begin{align*}
	& \max\, U(x,y) \\ 
	s.t.&\left\{\begin{aligned}
		2x+y&\leq 10 \\ x,y&\geq0
	\end{aligned}\right.
\end{align*}
При каких значениях параметрах $a$ 
\begin{enumerate}
	\item потребительская корзина состоит только из 
	второго товара
	\item содержит оба товара
\end{enumerate}
\end{exercise}

\begin{exercise}[*]%[Consumption--Leisure choice]
Экономический агент имеет два <<товара>>: <<отдых>> $l$ (leisure, в часах) и потребление $x$.
Пусть $w$ -- почасовая оплата и $P$ -- цена потребления. Агент располагает общим временем $H$,
которое он может тратить на работу и на отдых, и также имеет фиксированный доход $M$
(non-labor income). Функция полезности экономического агента $U(x,l)x+c\ln l$ ($c>0$). 
Рассмотрим задачу оптимизации
\begin{gather*}
	\max U(x,l) \\
	s.t.\left\{\begin{gathered}
		Px+wl\leq wH+M \\ 0\leq l\leq H \\ x\geq0
	\end{gathered}
	\right.
\end{gather*}
Найдите решение задачи оптимизации.
\end{exercise}

\begin{exercise}
Экономический агент потребляет два товара и его функция полезности равна
$U(x,y)=y+c\ln x$ ($c>0$). Цены на товары равны $P_1$ и $P_2$, доход равен $I$.
Сформулируйте задачу об оптимальной потребительской корзине и найдите её решение.
\end{exercise}

\subsection{Линейное программирование}

% !TEX root = exercise-book-opt.tex

\begin{exercise}
Рассмотрим задачи линейного программирования
\begin{align*}
	& \begin{gathered}
		\max(3x+5y) \\
		s.t.\left\{\begin{aligned}
			x+y &\leqslant5 \\ 2x+y &\leqslant8 \\ x,y&\geqslant0
		\end{aligned}\right.
	\end{gathered} &
	& \begin{gathered}
		\max(7x+4y) \\
		s.t.\left\{\begin{aligned}
			2x+5y &\leqslant30 \\ 2x+y &\leqslant14 \\ x,y&\geqslant0
		\end{aligned}\right.
	\end{gathered} \\
	& \begin{gathered}
		\max(4x+5y) \\
		s.t.\left\{\begin{aligned}
			x+3y &\leqslant15 \\ 4x+3y &\leqslant24 \\ x,y&\geqslant0
		\end{aligned}\right.
	\end{gathered} &
	& \begin{gathered}
		\max(8x+3y) \\
		s.t.\left\{\begin{aligned}
			2x+5y &\leqslant35 \\ 5x+3y &\leqslant40 \\ x,y&\geqslant0
		\end{aligned}\right.
	\end{gathered}
\end{align*}
\begin{enumerate} 
	\item Решите графическии (прямую) задачу
	\item Напишите и решите графически двойственную задачу.
\end{enumerate}
\end{exercise}

\begin{exercise}
Решите графически следующие задачи оптимизации
\begin{align*}
	& \begin{gathered}
		\max(5x+4y) \\
		s.t.\left\{\begin{aligned}
			x+3y &\leqslant18 \\ x+2y &\leqslant13 \\ 
			3x+2y &\leqslant27 \\ x,y&\geqslant0
		\end{aligned}\right.
	\end{gathered} &
	& \begin{gathered}
		\max(4x+3y) \\
		s.t.\left\{\begin{aligned}
			x+4y &\leqslant 28 \\ 2x+3y &\leqslant 26 \\ 
			x+y &\leqslant 11 \\ 2x+y &\leqslant 20 \\ x,y&\geqslant0
		\end{aligned}\right.
	\end{gathered} \\
	& \begin{gathered}
		\min(7x+6y) \\
		s.t.\left\{\begin{aligned}
			3x+y &\geqslant 9 \\ 4x+3y &\geqslant 22 \\ 
			x+3y &\leqslant 10 \\ x,y&\geqslant0
		\end{aligned}\right.
	\end{gathered} &
	& \begin{gathered}
		\min(2x+5y) \\
		s.t.\left\{\begin{aligned}
			3x+y &\geqslant 10 \\ 2x+y &\geqslant 8 \\ 
			x+3y &\geqslant 9 \\ x+6y &\geqslant 12 \\ x,y&\geqslant0
		\end{aligned}\right.
	\end{gathered}
\end{align*}
\end{exercise}

\begin{exercise}
Решите графически следующие задачи оптимизации
\begin{align*}
	& \begin{gathered}
		\max(5x-4y) \\
		s.t.\left\{\begin{aligned}
			5x+2y &\leqslant 16 \\ 2x-7y &\leqslant 22 \\ 
			-5x-y &\leqslant 19 \\ -x+3y &\leqslant 7
		\end{aligned}\right.
	\end{gathered} &
	& \begin{gathered}
		\max(-3x+2y) \\
		s.t.\left\{\begin{aligned}
			x+2y &\leqslant 9 \\ 3x-y &\leqslant 13 \\ 
			-2x-7y &\leqslant 22 \\ -5x+y &\leqslant 18 \\ 
			-x+4y &\leqslant 15
		\end{aligned}\right.
	\end{gathered} 
\end{align*}
\end{exercise}

% \begin{exercise}
% Рассмотрим задачу линейного программирование в матричном виде 
% \begin{gather*}
% 	\max(\vectf^\top\vectx) \\
% 	s.t.\left\{\begin{aligned}
% 		\matrixA\vectx & \leq\vectc \\ \vectx&\geq0
% 	\end{aligned}\right.
% \end{gather*}
% Для каждого из примеров решите 
% \end{exercise}

% \begin{exercise}
% Найдите решение задачи оптимизации
% \begin{gather*}
% 	\min (3x_1+4x_2) \\ 
% 	\left\{\begin{aligned} 
% 	& 3x_1+2x_2\geq 13 \\ & 5x_1+x_2\geq 10 \\ 
% 	& x_1+2x_2\geq 7 \\ & x_1,x_2\geq0
% 	\end{aligned}\right.
% \end{gather*}
% \end{exercise}

% \begin{exercise}
% Решите задачу оптимизации
% \begin{gather*}
% 	\max (3x_1+4x_2+2x_3+x_4)  \\
% 	\left\{\begin{aligned}
% 	2x_1+x_2+5x_3+5x_4 &\leq  40 \\
% 	x_1+2x_2+3x_3+2x_4 &\leq 30 \\
% 	x_1,x_2,x_3,x_4 & \geq 0
% 	\end{aligned}\right.
% \end{gather*}
% \begin{enumerate}
% 	\item симплекс-методом
% 	\item через двойственную задачу
% \end{enumerate}
% \end{exercise}

% \begin{exercise}
% Решите задачу оптимизации
% \begin{gather*}
% 	\max(x_1+2x_2+2x_3)\\
% 	\left\{\begin{aligned}
% 	   3x_1+x_2+x_3 &\leq 20 \\
% 		x_1+2x_2+2x_3 &\leq30 \\
% 		x_1, x_2, x_3 &\geq0
% 	\end{aligned}\right.
% \end{gather*}
% \begin{enumerate}
% 	\item симплекс-методом
% 	\item через двойственную задачу
% \end{enumerate}
% \end{exercise}

% \begin{exercise}
% Решите задачу оптимизации с использованием симплекс-метода
% \begin{gather*}
% 	\max(x_1+2x_2+2x_3+4x_4)\\
% 	\left\{\begin{aligned}
% 	   2x_1+2x_2+x_3+4x_4 &\leq 30 \\ 
% 		x_1+2x_2+2x_3+2x_4 &\leq 50 \\ 
% 		x_1, x_2, x_3, x_4 &\geq0
% 	 \end{aligned}\right.
% \end{gather*}
% \begin{enumerate}
% 	\item симплекс-методом
% 	\item через двойственную задачу
% \end{enumerate}
% \end{exercise}

% \begin{exercise}
% Решите задачу оптимизации
% \begin{gather*}
% 	\max(2x_1+2x_2+x_3+x_4)\\
% 	\left\{\begin{aligned}
% 		2x_1+2x_2+x_3+x_4 &\leq 20 \\ 
% 		x_1+x_2+2x_3+2x_4 &\leq 30 \\ 
% 		x_1, x_2, x_3, x_4 &\geq0
% 	\end{aligned}\right.
% \end{gather*}
% \begin{enumerate}
% 	\item симплекс-методом
% 	\item через двойственную задачу
% \end{enumerate}
% \end{exercise}

\begin{exercise}
Фирма производит четыре товара и использует для производства два ресурса.
Норма затрат ресурсов, количество ресурсов и прибыль от каждой единицы
товара приведены в таблице
\begin{center}
\begin{tabular}{|c|c|c|c|c||c|}
	\hline
	& Товар 1 & Товар 2 & Товар 3 & Товар 4 &  Количество \\
	& & & & &  ресурса \\
	\hline
	Ресурс 1 & 4 & 4 & 1 & 2 & 100\\ \hline
	Ресурс 2 & 5 & 3 & 2 & 1 & 150 \\ \hline \hline
	Цена & 20 & 12 & 4 & 2 &  \\ \hline
\end{tabular}
\end{center}
Предполагается, что нормы затрат постоянны и цены постоянны.

Постройте модель оптимизации производства и решите её численно 
(MS Excel/Python).
\end{exercise}

\begin{exercise}
Фирма производит три товара и использует для производства два ресурса.
Норма затрат ресурсов, количество ресурсов и прибыль от каждой единицы 
товара приведены в таблице
\begin{center}
\begin{tabular}{|c|c|c|c||c|}
	\hline 
	& Товар 1 & Товар 2 & Товар 3 & Количество \\
	& & & & ресурса \\
	\hline
	Ресурс 1 & 2 & 1 & 5 & 100 \\ \hline
	Ресурс 2 & 4 & 2 & 3 & 120 \\ \hline \hline
	Цена & 3 & 8 & 2 & \\ \hline
\end{tabular}
\end{center}
Предполагается, что нормы затрат постоянны и цены постоянны.

Постройте модель оптимизации производства и решите её численно 
(MS Excel/Python).
\end{exercise}

\begin{exercise}
Фирма производит три товара и использует для производства два ресурса.
Норма затрат ресурсов, количество ресурсов и прибыль от каждой единицы
товара приведены в таблице
\begin{center}
\begin{tabular}{|c|c|c|c||c|}
	\hline
	& Товар 1 & Товар 2 & Товар 3 & Количество  \\
	& & & & ресурса \\
	\hline
	Ресурс 1 & 2 & 1 & 5 & 100 \\ \hline
	Ресурс 2 & 5 & 2 & 5 & 220 \\ \hline \hline
	Цена & 3 & 8 & 2 & \\ \hline
\end{tabular}
\end{center}
Предполагается, что нормы затрат постоянны и цены постоянны.

Постройте модель оптимизации производства и решите её численно 
(MS Excel/Python).
\end{exercise}
	

% \begin{exercise}
% Фирма производит три товара и использует для производства два ресурса.
% По плану первого товара нужно произвести не менее 100 единиц, второго --
% не менее 120, третьего -- не менее 150 ед.
% Норма затрат ресурсов и цена на ресурсы приведены в таблице
% \begin{center}
%  \begin{tabular}{|c|c|c||c|}
%   \hline 
%   & Ресурс 1 & Ресурс 2 & План \\
%   \hline
%   Товар 1 & 2 & 1 &  100 \\ \hline
%   Товар 2 & 4 & 3 &  120 \\ \hline
%   Товар 3 & 3 & 5 &  150 \\ \hline \hline
%   Цена & 3 & 6 & \\ \hline
%  \end{tabular}
% \end{center}
% Предполагается, что нормы затрат постоянны и цены экзогенны.
% \begin{enumerate}[i)]
%  \item Постройте модель оптимизации затрат ресурсов.
%  \item Постройте двойственную задачу.
%  \item Найдите оптимального количество ресурсов.
%  \item Найдите решение двойственной задачи и дайте
%  экономическую интерпретацию этого решения.
% \end{enumerate}
% \end{exercise}

\begin{exercise}%[\textbf{11 баллов}]
Фирма <<Московия>> заключила контракт с компанией АЛРОСА на покупку
промышленного золота для его реализации в пяти городах в объеме:
Самара -- 80 кг, Москва -- 260 кг, Ростов-на-Дону -- 100 кг,
Санкт-Петербург -- 140 кг, Нижний Новгород -- 120 кг. Компания
АЛРОСА располагает тремя месторождениями: <<Мирное>>, <<Удачный>> и
<<Полевое>>, которые планируют за год выработать соответственно
200, 250 и 250 кг золота.

Постройте модель оптимизации фрахта специализированного транспорта,
обеспечивающего полное удовлетворение заявок покупателя, при
заданной системе тарифов (на 1 кг)
\begin{center}\small
	\begin{tabular}{|c|c|c|c|c|c|}
	\hline
	% after \\: \hline or \cline{col1-col2} \cline{col3-col4} ...
	& Самара & Москва & Ростов-на-Дону & С.-Пб. & Н. Новгород \\ \hline
	<<Мирное>> & 7 & 9 & 15 & 4 & 18\\
	<<Удачный>> & 13 & 25 & 8 & 15 & 5 \\
	<<Полевое>> & 5 & 11 & 6 & 20 & 12\\
	\hline
	\end{tabular}
\end{center}
Найдите \textbf{численно} оптимальное решение.
\end{exercise}

\begin{exercise}
Московский филиал <<The Coca-Cola Company>>, выпускающей напитки
приблизительного равного спроса (Sprite, Coca-Cola, Fanta),
складируемых в разных местах, должен поставить свою продукцию в
четыре крупных супермаркета: <<Ашан>>, <<Карусель>>, <<Седьмой
Континент>> и <<Арбатский>>. Каждая упаковка содержит 12 банок
емкостью 0.33 литра. Тарифы на доставку, объемы запасов и заказы на
продукцию приведены в таблице.
\begin{center}\footnotesize
	\begin{tabular}{|c|c|c|c|c|c|}
	\hline
	% after \\: \hline or \cline{col1-col2} \cline{col3-col4} ...
	& \multicolumn{4}{|c|}{Супермаркеты} & \\ \hline
	Склады & Ашан & Карусель & Перекрёсток& Дикси &
	Запасы, уп.
	\\ \hline
	Coca-Cola & 6 & 4 & 9 & 5 & 400 \\
	Sprite & 5 & 7 & 8 & 6 & 300 \\
	Fanta & 9 & 4 & 6 & 7 & 200 \\ \hline
	Заказы, уп. & 150 & 250 & 150 & 350 &  \\
	\hline
	\end{tabular}
\end{center}
Постройте оптимизационную модель плана поставок напитков в
супермаркеты. Найдите \textbf{численно} оптимальное решение.
\end{exercise}

\begin{exercise}
Коммерческое предприятие реализует три группы товаров A, B и C.
Плановые нормативы затрат ресурсов (на 1 тыс рублей товарооборота),
доход от продажи товаров (на 1 тыс. рублей товарооборота)
приведены в таблице
\begin{center}\small
	\begin{tabular}{|l|c|c|c|c|}
	\hline
	& \multicolumn{3}{|c|}{Нормы затрат} & \\ \hline
	% after \\: \hline or \cline{col1-col2} \cline{col3-col4} ...
	Ресурсы & A & B & C & Объем ресурсов \\ \hline
	Рабочее время продавцов & 0.1 & 3 & 0.4 & 1100 \\
	Площадь торговых залов & 0.05 & 0.2 & 0.02 & 120 \\
	Площадь складских помещений & 3 & 0.02 & 2 & 8000 \\ \hline
	Доход & 3 & 1 & 4 &  \\
	\hline
	\end{tabular}
\end{center}
Постройте модель оптимизации для получения максимального дохода.
Найдите \textbf{численно} оптимальное решение.
\end{exercise}

% \begin{exercise}
% Для поддержания нормальной жизнедеятельности человеку ежедневно
% необходимо потреблять 118г белков, 56г жиров, 500г углеводов, 8г
% минеральных солей. Количество питательных веществ, содержащихся в
% 1кг имеющихся в магазине продуктов питания, а также их стоимость
% приведены в таблице
% \begin{center}
% {\small
% \begin{tabular}{|l|c|c|c|c|c|c|c|c|}
%   \hline
%    & \multicolumn{7}{|c|}{Содержание в 1 кг продуктов} &  \\ \hline
%   % after \\: \hline or \cline{col1-col2} \cline{col3-col4} ...
%    & мясо & рыба & молоко & масло & сыр & крупа & картофель & Нормы
%    \\ \hline
%   Белки, г & 180 & 190 & 30 & 70 & 260 & 130 & 21 & 118 \\
%   Жиры, г & 20 & 3 & 40 & 865 & 310 & 30 & 2 & 56 \\
%   Углеводы, г & 0 & 0 & 50 & 6 & 20 & 650 & 200 & 500 \\
%   Мин. соли, г & 9 & 10 & 7 & 12 & 60 & 20 & 70 & 8 \\ \hline
%   Стоимость, кг & 1.9 & 1.0 & 0.28 & 3.4 & 2.9 & 0.56 & 0.1 &  \\
%   \hline
% \end{tabular}
% }
% \end{center}
% Требуется составить модель оптимизации суточного рациона,
% содержащего не менее суточной потребности человека в белках, жирах,
% углеводах, минеральных солях и обеспечивающего минимальную стоимость
% продуктов.
% \end{exercise}

\begin{exercise}
Три нефтеперерабатывающих завода с (ежедневной) производительностью
6, 5 и 8 млн.т бензина снабжают три бензохранилища, (ежедневно)
потребность которых составляет  4, 8 и 7 млн. т бензина соответственно.
Бензин транспортируется в бензохранилища по бензопроводу. Стоимость
транспортировки составляет 0.3 руб за 1000 т на один км длины бензопровода.
В таблице приведены расстояния в км между заводами и хранилищами.
Отметим, что первый нефтеперерабатывающий завод не связан бензопроводом
с третьим бензохранилищем.
\begin{center}%\small
	\begin{tabular}{|c|c|c|c|c|} \hline
	& \multicolumn{3}{|c|}{Хранилища} & \\ \hline
 	Завод & 1 & 2 & 3 & Объем \\ \hline
 	1 & 120 & 180 & --- & 6 \\ \hline
	2 & 300 & 100 & 80 & 5 \\ \hline
	3 & 200 & 250 & 120 & 8 \\ \hline
	Вместимость & 4& 8 & 7 & \\ %\hline
	хранилища & & & & \\ \hline
	\end{tabular}
\end{center}
Постройте оптимизационную модель транспортировки бензина.
Найдите \textbf{численно} оптимальное решение.
\end{exercise}

\section{Введение в теорию игр}

\input{game-theory.tex}

\appendix

\section{Приложение}

% !TEX root = exercise-book-opt.tex

\subsection{Симметричные матрицы}

Пусть \(\matrixA\) -- (\(n\times n\)) симметричная матрица.

\begin{teorema}[Критерий Сильвестра]\label{SylvesterCiterion}
Пусть $\matrixA$ -- симметричная матрица и
$\Delta_1,\ldots,\Delta_n$ последовательность ее угловых миноров:
\begin{align*}
	\Delta_1&=a_{11} & \Delta_2&=\det\begin{pmatrix} a_{11} & a_{12} \\ a_{21} & a_{22} \end{pmatrix} & 
	&\ldots & \Delta_n&=\det\matrixA
\end{align*}
Тогда
\begin{enumerate}
	\item $\matrixA>0\iff\Delta_i>0$, $i=1,\ldots,n$.
	\item $\matrixA<0\iff(-1)^i\Delta_i>0$, $i=1,\ldots,n$.
	\item если знаки миноров не удовлетворяют предыдущим пунктам, 
	то матрица не знакоопределена
\end{enumerate}  
\end{teorema}

\begin{propuesta}\label{2times2definitness}
Пусть $\matrixA$ -- симметричная $2\times 2$ 
Тогда
\begin{align*}
	\matrixA\geq0 &\iff \begin{matrix} a_{11} \\ a_{22} \end{matrix} \geq0,\;\det\matrixA\geq0\\
	\matrixA\leq0  &\iff \begin{matrix} a_{11} \\ a_{22} \end{matrix} \leq0,\;\det\matrixA\geq0.
\end{align*}    
\end{propuesta}

Для  $3\times3$ матрицы обозначим центральные миноры 
\begin{align*}
	\Minor_{(12)}&=\det\begin{pmatrix} a_{11} & a_{12} \\ a_{21} & a_{22} \end{pmatrix} \\
	\Minor_{(13)}&=\det\begin{pmatrix} a_{11} & a_{13} \\ a_{31} & a_{33} \end{pmatrix} \\
	\Minor_{(23)}&=\det\begin{pmatrix} a_{22} & a_{23} \\ a_{32} & a_{33} \end{pmatrix} 
\end{align*}

\begin{propuesta}\label{3times3definitness}
Пусть $\matrixA$ -- симметричная $3\times 3$. Тогда
\begin{align*}
	\matrixA\geq 0 &\iff \begin{matrix} a_{11} \\ a_{22} \\ a_{33} \end{matrix}\geq0,\;
	\begin{matrix} \Minor_{(12)} \\ \Minor_{(13)} \\ \Minor_{(23)} \end{matrix}\geq0,\;
	\det\matrixA\geq0\\
	\matrixA\leq 0 &\iff \begin{matrix} a_{11} \\ a_{22} \\ a_{33} \end{matrix}\leq0,\;
	\begin{matrix} \Minor_{(12)} \\ \Minor_{(13)} \\ \Minor_{(23)} \end{matrix}\geq0,\;
	\det\matrixA\leq0.
\end{align*}
\end{propuesta}

\subsection{Выпуклые функции}

Пусть числовая функция \(f\) определена на % выпуклом множестве 
\(\Domain(f)\subset \R^n\)

\begin{teorema}
Дважды непрерывно дифференцируемая функция $f$ выпукла $\iff$ 
$\Hessian_f(\vectx)\geq0$ %(как симметричная матрица) 
для всех $\vectx\in\Domain(f)$.

Если $\Hessian_f(\vectx)>0$ для всех $\vectx\in\Domain(f)$, 
то функция строго выпукла на $\Domain(f)$.
\end{teorema}

\begin{col}
Дважды непрерывно дифференцируемая функция $f$ вогнута $\iff$ 
$\Hessian_f(\vectx)\leq0$ % (как симметричная матрица) 
для всех $\vectx\in\Domain(f)$.

Если $\Hessian_f(\vectx)<0$ для всех $\vectx\in\Domain(f)$, 
то функция строго вогнута
на $\Domain(f)$.
\end{col}

\begin{remark}
Знак гессиана проверяем по критерию Сильвестра 
\ref{SylvesterCiterion} или
используем Предложения \ref{2times2definitness}, \ref{3times3definitness}
\end{remark}


\subsection{Функция Лагранжа для ограничений равенства}

Рассмотрим задачи оптимизации с ограничениями равенства

\begin{align*}
	& \begin{gathered}
		\max f(\vectx) \\
		s.t.\left\{\begin{aligned}
			g_1(\vectx)&=c_1 \\ &\vdots \\ g_k(\vectx)&=c_k
		\end{aligned}\right.
	\end{gathered} &
	& \begin{gathered}
		\min f(\vectx) \\
		s.t.\left\{\begin{aligned}
			g_1(\vectx)&=c_1 \\ &\vdots \\ g_k(\vectx)&=c_k
		\end{aligned}\right.
	\end{gathered}
\end{align*}
Функция Лагранжа для этих задач
\[
	\Lagrange(\vectx,\vectlambda)=f(\vectx)-\sum_{j=1}^k \lambda_jg_j(\vectx)
\]
\textbf{Необходимые условия} (локального) условного экстремума
\[
	\left\{\begin{aligned}
		\Lagrange'_{x_i}&=0 & i&=1,\ldots,n\\
		g_j(\vectx) &= c_j & j&=1,\ldots,k
	\end{aligned}\right.
\]
Гессиан для функции Лагранжа (симметричная матрица)
\[
	\underset{(n+k)\times(n+k)}{\BordHessian_\Lagrange}=\begin{pmatrix}
		\frac{\partial^2 \Lagrange}{\partial x_i\partial x_j} & | &\frac{\partial^2 \Lagrange}{\partial x_i\partial \lambda_l} \\
		-- & + & -- \\
		\frac{\partial^2 \Lagrange}{\partial \lambda_s\partial x_j} & | &
		\frac{\partial^2 \Lagrange}{\partial \lambda_s\partial \lambda_l}
   \end{pmatrix}
\]
Из определения функции Лагранжа
\begin{itemize}
	\item \(\frac{\partial^2 \Lagrange}{\partial \lambda_s\partial \lambda_l}=0\)
	\item \(\frac{\partial^2 \Lagrange}{\partial \lambda_s\partial x_j}=-\frac{\partial g_s}{\partial x_j}\)
\end{itemize}
Явный вид гессиана
\begin{equation}\label{BorderedHessian}
	\underset{(n+k)\times(n+k)}{\BordHessian_\Lagrange}=
%    \begin{pmatrix}
%      \frac{\partial^2 \Lagrange}{\partial x_i\partial x_j} & | &\frac{\partial^2 \Lagrange}{\partial x_i\partial \lambda_l} \\
%      -- & + & -- \\
%      \frac{\partial^2 \Lagrange}{\partial \lambda_s\partial x_j} & | &
%      \frac{\partial^2 \Lagrange}{\partial \lambda_s\partial \lambda_l}
%    \end{pmatrix}=\\
	\begin{pmatrix} 
		\frac{\partial^2 \Lagrange}{\partial x_1\partial x_1} &  \cdots &
		\frac{\partial^2 \Lagrange}{\partial x_1\partial x_n} & -\frac{\partial g_1}{\partial x_1} & \cdots & 
		-\frac{\partial g_k}{\partial x_1} \\ 
		\frac{\partial^2 \Lagrange}{\partial x_2\partial x_1} &  \cdots &
		\frac{\partial^2 \Lagrange}{\partial x_2\partial x_n} & -\frac{\partial g_1}{\partial x_2} & \cdots & 
		-\frac{\partial g_k}{\partial x_2}\\
		\vdots & \ddots & \vdots & \vdots & \ddots & \vdots \\
		\frac{\partial^2 \Lagrange}{\partial x_n\partial x_1} & \cdots &
		\frac{\partial^2 \Lagrange}{\partial x_n\partial x_n}& -\frac{\partial g_1}{\partial x_n} & \cdots & 
		-\frac{\partial g_k}{\partial x_n}\\
		-\frac{\partial g_1}{\partial x_1} & \cdots &
		-\frac{\partial g_1}{\partial x_n} & 0 & \cdots & 0\\
		\vdots & \ddots & \vdots & \vdots & \ddots & \vdots\\
		-\frac{\partial g_k}{\partial x_1} &  \cdots &
		-\frac{\partial g_k}{\partial x_n} & 0 & \cdots & 0\\
	\end{pmatrix}
\end{equation}
Пусть $\Minor_{i}$ ($i=1,...,n+k$) -- главный минор матрицы $\BordHessian_\Lagrange$, 
образованный строками и столбцами с индексами $i,i+1,...,n+k$.

\begin{teorema}[Достаточные условия минимума]\label{EqualityConstraintMinSufficientCondition}
Пусть в точке $\hat{\vectx}$  ранг матрицы $(\frac{\partial g_j}{\partial x_i})$ максимален и 
эта точка удовлетворяет необходимым условия экстремума.\\
Тогда достаточным условием локального минимума является выполнение неравенств
\begin{equation}\label{EqualityConstraintMinSignSufficientCondition}
	(-1)^k\Minor_1(\hat{\vectx}),\ldots,(-1)^k\Minor_{n-k}(\hat{\vectx})>0.
\end{equation}
\end{teorema}
\begin{remark}
Условие \eqref{EqualityConstraintMinSignSufficientCondition} означает, что
все миноры $\Minor_1,\ldots,\Minor_{n-k}$ имеют знак $(-1)^k$.
\end{remark}

\begin{teorema}[Достаточные условия максимума]\label{EqualityConstraintMaxSufficientCondition}
Пусть в точке $\hat{\vectx}$  ранг матрицы $(\frac{\partial g_j}{\partial x_i})$ максимален и 
эта точка удовлетворяет необходимым условия экстремума.\\
Тогда достаточным условием наличия максимума является выполнение неравенств
\begin{align}\label{EqualityConstraintMaxSignSufficientCondition}
	(-1)^n(-1)^{i-1}\Minor_i(\hat{\vectx})&>0 & i&=1,\ldots,n-k.
\end{align}
\end{teorema}
\begin{remark}
Условие \eqref{EqualityConstraintMaxSignSufficientCondition} означает чередование знаков
в последовательности миноров $\Minor_1,\ldots,\Minor_{n-k}$, начиная со знака $(-1)^n$.
\end{remark}
	

\end{document}